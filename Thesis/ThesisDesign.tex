\documentclass[a4paper,pdf]{article} % gebruik acm style voor je scriptie: [sigconf, format=acmsmall, screen=true, review=false]{acmart} 

 
\usepackage{hyperref}
\usepackage{pdfpages} % http://mirror.unl.edu/ctan/macros/latex/contrib/pdfpages/pdfpages.pdf
\usepackage{booktabs} 
\usepackage[utf8]{inputenc}
\usepackage{amsmath}
\usepackage{graphicx}
\usepackage[colorinlistoftodos]{todonotes} % handig voor commentaar: gebruik \todo{}, zie ftp://ftp.fu-berlin.de/tex/CTAN/macros/latex/contrib/todonotes/todonotes.pdf
\usepackage{listings}
\usepackage{pdfpages}
\usepackage{tcolorbox}
\usepackage{float}
\usepackage{caption}
\usepackage{subcaption}


% when writing in Dutch
%\usepackage[dutch]{babel}
%\selectlanguage{dutch}

\begin{document}
\title{Fake news: an algorithmic approach} % your title
\author{Martijn Schouten}

\maketitle

%\todototoc
%\listoftodos
%\tableofcontents

%\begin{abstract}
%My abstract
%\end{abstract}


\section{Personal details}

\begin{description}
 \item[My email] \url{mailto:martijn.schouten@student.uva.nl }
 \item[My supervisors email] \url{mailto:maartenmarx@uva.nl }
 \item[The wiki on my GitHub account] \url{https://github.com/MeMartijn/BachelorThesis/wiki}
 \end{description} 

\section{Research question}

%2. A clearly defined research problem and corresponding subquestions **(20)**
%	* Can the problem be answered?
%	* Do answers to the subquestions indeed help in an understanding of the research problem or even in solving the research problem?
%	* Are the subquestions detailed enough?
The following research question is defined: \textit{how well can state-of-the-art natural language processing techniques in combination with machine learning algorithms classify fake news?}

For this research question, the following subquestions will be answered:
\begin{itemize}
  \item Q1: How can fake news be defined and characterized?
  \item Q2: What new ways of word- or sentence embeddings can be used for encoding plain text?
  \item Q3: What is the performance of combinations of these novel embedding techniques with machine learning algorithms?
  \item Q4: To what extent can performance of fake news classifiers be improved with increased amounts of raw data?
\end{itemize}


%  \section{Related Literature}
% Overview of the state of the art of the literature **(20)**
%	* One expects that the research problem is grounded in the literature and that each subquestion or field has a small section of relevant literature.
%	* All parts of the thesis should be grounded in or at least connected to  the literature.
\section{Related Work}

\subsection{RQ1}
Fake news as a term only caught public attention starting from the end of 2016, during the Presidential Elections of the United States \cite{googletrends2019}.   

\subsection{RQ2}
In the last couple of years, using transfer learning for natural language processing has given promisable results. The following sentence embeddings will be used to detect fake news:

\begin{itemize}
    \item Bag of Words as a baseline for performance of non-pretrained embeddings;
    \item Facebook's InferSent \cite{conneau2017};
    \item ELMo from the Allen Institute for Artificial Intelligence \cite{peters2018};
    \item OpenAI's GPT-2 \cite{radford2019};
    \item Transformer-XL \cite{dai2019};
    \item Microsoft's MT-DNN  \cite{liu2019};
    \item and Google's BERT \cite{devlin2018}.
\end{itemize}

\subsection{RQ3}
Aligned with the original research on this dataset by Wang \cite{wang2018}, the following machine learning algorithms will be used to test the applicability of the abovementioned embedding techniques: 
\begin{itemize}
    \item SVMs;
    \item Logistic regression;
    \item Bi-LSTMs;
    \item CNNs.
\end{itemize}
	
% \section{Methodology}
%Methodology **(20)**
%	5. Do I get a clear picture of the used resources?
%		5. E.g., for data, do I get a clear picture of the data, its state, its availability, how much it is, how dirty, how much work to process, etc, etc.
%	6. Are the methods which will be used described in enough detail, so that I can picture what will be done exactly? 
%	7. Is the evaluation appropriate? That is, do I understand how each subquestion is answered by the evaluation? 



% \section{Risk assessment}
%Risk assessment **(10)**
%	* Is it complete? Is is realistic? Is the backup plan executable?
	
	
% \section{Project plan}
%Project plan  **(20)**
%	* Is it complete? (I.e., every part of the work covered.)
%	* Is it realistic?
%	* Does it give a clear picture of what will be done when? 
%	* Is it possible to evaluate whether the student is on schedule at any point in time?


 

 
% your refs

\bibliographystyle{plain}
\bibliography{MyThesis}
\end{document}