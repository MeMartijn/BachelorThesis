\section{Related Work}
\subsection{Automatic fake news detection}
In past research, there have been several attempts to create classifiers for automatic detection of lies and fake news. 
Wang used both neural and non-neural classifiers to classify statements from the Liar dataset into 6 possible gradations of truthfulness. 
Furthermore, he added speaker metadata to improve the result of his classifications.
Both with and without introducing speaker metadata, the best performing architecture was found to be a convolutional neural network. 
With an accuracy of 27\% without, and 27,4\% with metadata on the test set, Wang was able to perform 6,2\% and 6,6\% better than the majority baseline of 20,8\%\cite{wang2018}.

From the same dataset, Khurana extracted linguistic features such as n-grams, sentiment, number of capital letters and POS tags to classify the data into 3 labels instead of the original 6 labels. 
For classification, she used a set of non-neural classifiers. 
Her best performing classifier, using gradient boosting, obtained an accuracy of 49,03\%, which performed around 5\% better than the majority baseline of 44,28\% \cite{khurana2017}.

The British factchecking organization Full Fact has developed an architecture that is able to monitor and factcheck statements from the British Parliament and major media outlets in the United Kingdom. 
One of its uses is automatically factchecking the accuracy of statistical claims made by members of the Parliament \cite{babakar2016}.
For detecting factual claims from texts, the organization uses InferSent, which is a way of transfer learning on sentence level that has been proved to perform well for the use case of Full Fact \cite{pydata2018}.

Various tools with regards to fake news detection and research are also available. 
Faker Fact is a tool which can classify texts into a set of categories ranging from satire to agenda-driven, the former identifying humorous intent, the latter identifying manipulation \cite{fakerfact}. 
For tracking online misinformation, the Observatory on Social Media created Hoaxy. 
Hoaxy allows for the visualization of the spread of unverified claims through Twitter networks \cite{shao2016}.

\subsection{Pre-trained textual embeddings}
Traditionally, feature representation for text classification is often based on the bag-of-words model, containing linguistic patterns as features, such as unigrams, bigrams or n-grams.
However, these approaches completely ignore contextual information or word order in texts, and are unable to capture semantics of words. 
As a result, classifiers may be unable to correctly identify patterns, affecting the classification accuracy \cite{lai2015}.

As an answer to these problems, pre-trained text embeddings have been rising in popularity, both in use and in research.
Before classification is possible, text data needs to be transformed into numbers to be able to be interpreted by classification algorithms.
Fundamentally, text embeddings are vector representations of linguistic structures, allowing for usage of text in classifiers. 
The process of turning words into these embeddings is typically powered by statistics gathered from large unlabeled corpora of text data \cite{mikolov2017}.

In 2017, Vaswani et al. proposed a novel architecture for embedding raw textual data called the Transformer. 
With the main aim originally being translating one sentence from one language to another, Transformers are based on an encoder-decoder model.
These models take the sequence of input words and convert it to an intermediate representation, after which a decoder creates an output sequence.

The main strength of the Transformer architecture is its focus on attention to create the intermediate representation.
The encoder receives a sequence of inputs, and reads it at once, as opposed to sequentially (either from left to right or from right to left, as humans do). 
This allows the encoder to learn the context of a word based on all of its surrounding text \cite{vaswani2017}, as opposed to traditional vector representation techniques that only allow a single context-independent representation for each word \cite{peters2018}.
As shown by the Bidirectional Encoder Representations from Transformer (BERT) model by Devlin et al., these techniques have beaten existing benchmarks in natural language processing, underlining the importance of context in textual data \cite{devlin2018}.

\subsection{Pooling}
Most non-neural classifiers need data in a two-dimensional uniform shape to be able to perform calculations. 
In the case of raw text data, sentences in datasets often have variable word lengths, resulting in a different vector length when turning the texts into a vector representation.
Furthermore, when dealing with vector representations of words, the shape of a statement is turned three-dimensional instead of the needed two-dimensionality for non-neural classifications.
To turn the vector representations into a uniform length, we can either cut off the vectors at a fixed length (\textit{padding}), or we can perform calculations to reduce the length and dimensions of the vectors (\textit{pooling}).

In computer vision, feature pooling is often used to reduce noise in data. 
The goal of this step is to transform joint feature representations into a new, more usable one that preserves important information while discarding irrelevant details.
Pooling techniques such as max pooling and average pooling perform mathematical operations to reduce several numbers into one \cite{boureau2010}. 
In the case of transforming the shape of data, we can reduce vectors to the smallest vector in the dataset to create a uniform shape.

Scherer et al. compared performance of two pooling operations on a convolutional neural network architecture. 
The first pooling method extracted maximums and the second one was primarily based on working with averages.
They have shown that a max pooling operation is vastly superior for capturing invariances in image-like data \cite{scherer2010}.

Shen et al. noted that in text classification, only a small number of key words contribute to the final prediction.
As a result, simple pooling operations are surprisingly effective for representing documents \cite{shen2018}.  
Lai et al., Hu et al. and Zhang et al. use a max pooling layer in a (recurrent) convolutional neural network for identifying key features in text classification \cite{lai2015,hu2014,zhang2015}. 
For text classification, max pooling strategies seem to be the most popular. 

\subsection{Padding}
When padding a sequence, a list of sequences is transformed to a specific length. 
Sequences longer than the desired length will be truncated to fit the requirement, while sequences shorter than the desired length will be padded by a specified value \cite{keraspad}. 
To fill the sequences, a value of zero is often used. Hu et al. also use zero values for padding their sequences \cite{hu2014}. 

Apart from controlling the size of the feature dimension, padding has other uses as well. 
Simard et al. make use of sequence padding for convolutional neural networks to center feature units, and concluded it did not impact the performance of the classifier significantly \cite{simard2003}. 
Wen et al. apply padding to convolutional network models to prevent dimension loss \cite{wen2018}. 

\subsection{Neural text classifiers}
Wang has shown that neural networks perform slightly better on classifying fake news than non-neural classifiers. 
In his research, he compared accuracies on support vector machines, logistic regressions, bidirectional LSTMs and convolutional neural networks with each other.
With his 6 label classification, his support vector machine implementation was the best performing linear classifier, but the performance was slightly worse than the best performing neural network (25,8\% for the former, and 26\% for the latter) \cite{wang2018}.

Wang used two neural network architectures both well known for their robustness and performance when it comes to text classification. 
The first model, the bidirectional Long Short Term Memory (LSTM) network, is specifically tailored at keeping track of information for a long period of time. 
This makes the model able to keep track of the context in a more intelligent way when compared to a standard non-neural classification algorithm \cite{olah2015}. 

The second architecture, the convolutional neural network, applies a set of convolving filters in its layers that are applied to local features. 
These models are shown to be effective in numerous natural language processing applications, such as semantic parsing, search query retrieval, sentence modeling and other traditional NLP tasks \cite{kim2014}.  